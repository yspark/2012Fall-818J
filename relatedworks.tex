\section{Related Works}
\label{sec:relatedworks}

\cite{Bavier98predictingmpeg}\cite{DVFS} indicate that MPEG\cite{MPEG} decoding does not consume a constant processing time, mainly because of the facts that 1) a given MPEG video stream contains different frame types and 2) there exists wide variations in changes between frames. 

\cite{MPEG} MPEG terminology. \term{I-frame}, \term{P-frame}, \term{B-frame}. (from DVFS 2.2). 
The authors empirically studied the relationship between the decoding time and the data size of each frame.  (from DVFS 2.3). 

\cite{DVFS}  predicts the amount of computational workload of ARM processor based on the type of each incoming frame.
IDCT and Reconstruction take more than half of CPU time. 
IDCT is CPU intensive and its computation time is dependent on frmae type.



\cite{Zhang:2009} is the first to indicate that security-sensitive information can be revealed by standard Unix \term{proc} file system. 
The authors focus on the correlation between the program's system call and its ESP (stack pointer) value for keystroke sniffing, 

while \ciet{jana:memento} exploits the dynamics of memory footprints, that is, sequences of snapshots of the target program's data resident size (DRS).
Those works exploits the fact that any adversary 

and more citations show a process without admin privilege can monitor other process' information through \term{proc}. 

shows CPU usage pattern can be used for side-channel attack. 

shows a person's preference on movies can reveal sensitive information such as ....
