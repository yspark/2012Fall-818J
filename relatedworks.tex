\section{Related Works}
\label{sec:relatedworks}

\cite{Zhang:2009} is the first to indicate that security-sensitive information can be revealed by standard Unix \term{proc} file system. 
The attack in this work takes advantage of the stack information of a target process contained in the proc file system. 
The authors focus on the correlation between the system call of the target process and its ESP (stack pointer) values. 
They show that from a keystroke event can be identified the ESP content if they trigger system calls.
Then, they perform the timing attack to infer the characters the victim input. 
Our work also exploits the similar side-channel. 
In our work, we use the standard Unix commands, \term{ps} or \term{top}, in order to measure the dynamics of CPU usage of a victim's process. 
An adversary is able to use those command to get information of other processes, such as process ID, CPU usage, and memory usage, without any permission. 

\cite{jana:memento} is also motivated by the fact that the proc file system can reveal security-sensitive information of other process. 
The authors exploit the dynamics of memory footprints, that is, sequences of snapshots of the target program's Data Resident Size (DRS).
By monitoring memory footprints of victim's web browser process, the authors can infer which site the victim has visited. 
Further the authors extended the attack onto Android platform, where any Android app can use the standard Unix proc facility without any permission or phone owner's consent. 

\cite{Saponas07devicesthat} shows the study of \term{Slingbox}\cite{slingbox}, one of the next-generation wireless video multimedia systems for the home. 
The authors predict the content that is being transferred on Slingbox network by monitoring the transfer rate at which multimedia data is being sent from one device to the other. 
They take advantage of the facts that variable bit rate encoding is widely used for network efficiency and each multimedia content has unique throughput trace. 
However, it has one crucial limitation: Only Slingbox traffic should exist on the network so that the throughput of Slingbox content can be measured exactly. 
Our work does not have this limitation since we are targeting a specific application process such as Netflix app.
Even if a victim is running multiple processes on his device, we can target a specific process and get CPU usage statistics. 

\cite{Bavier98predictingmpeg}\cite{DVFS} indicate that MPEG decoding does not consume a constant processing time, mainly because of the facts that 1) a given MPEG video stream contains different frame types and 2) there exists wide variations in changes between frames. 
The authors exploit this observation in order to predict the amount of computational workload of the processor based on type and size of each incoming MPEG frame.
Our work takes advantage of these preceding researches to build a fundamental base: Each MPEG stream has different constitution of frames and scenes, and therefore it has a different, and possibly unique, CPU usage statistics. 
So in this work, we use CPU usage statistics for our side-channel attack to predict the content of MPEG video stream. 



\begin{comment}

\cite{MPEG} MPEG terminology. \term{I-frame}, \term{P-frame}, \term{B-frame}. (from DVFS 2.2). 
The authors empirically studied the relationship between the decoding time and the data size of each frame.  (from DVFS 2.3). 

\cite{DVFS}  predicts the amount of computational workload of ARM processor based on the type of each incoming frame.
IDCT and Reconstruction take more than half of CPU time. 
IDCT is CPU intensive and its computation time is dependent on frmae type.


and more citations show a process without admin privilege can monitor other process' information through \term{proc}. 

shows CPU usage pattern can be used for side-channel attack. 

shows a person's preference on movies can reveal sensitive information such as ....

\end{comment}