\section{Introduction}
\label{sec:introduction}
We present a new trojan-like passive side-channel attack on multimedia. Our goal is to reveal a title of a movie a user is watching by monitoring Central Processing Unit (CPU) usage statistics of the application which renders the movie. We exploit two distinct features of multimedia streams: variable bit rates of streams, and cpu-intensive workloads of decoding process. Multimedia streams are encoded in variable bit rates because each video frame of the streams contains varying visual information itself and visual differences to its reference video frames, which allows the video codec to exploit spatial and temporal locality, respectively. These encoding factors will affect the size of frame and GOP, and, as a result, an amount of CPU operations to decode a movie in a client-side because multimedia decoding is very CPU-intensive process. The decoding process may generate a movie-specific CPU usage patterns, hopefull, unique ones. Therefore, we expect that, with high probability, the attacker can identify a title of a movie by monitoring CPU usage statistics of an multimedia application.\\
\indent We will experimentally show the feasibility and the practicability of our attack. Based on a pre-built CPU usage statistics database for a set of movies, a title of an arbitraty movie randomly selected from the set will be identified by matching an monitored CPU usage pattern againt our database. Subsequence matching in time-series database can be employed to achieve our goal.  Our experiments will be a two-fold: a laptop with Linux operating systems, and a mobile device with Android platform. In the Linux environment system, the attack on a native media player and a web-based video player (i.e. Microsoft Silverlight on Netflix) will be assessed. In the Android-based mobile platform, the Netflix application will be used to evaluate our attack.\\
\indent The proposed attack will be serious threats to users' sensitive personal information(, revealing e.g. political inclination, religious interest and sexual orientation). The threats are realistic as millions of users are watching movies on their mobile devices due to the advent of numerous web-based multimedia contents providers such as Netflix, Hulu. Also, one of the most wide-spread mobile platform, Android, allows even an unprivileged process or application to monitor CPU usage of other processes, which makes the threats more realistic. 